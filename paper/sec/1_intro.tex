\section{Introduction}
Agents that operate full desktops or browsers still fail on small targets, animated widgets, and transient lag. Humans compensate using \emph{motor micro-signals} visible in ongoing cursor trajectories hundreds of milliseconds before a click. Two strands justify treating cursor dynamics as a first-class input: (1) \emph{cursor $\leftrightarrow$ gaze/attention} correlation on SERPs and browsing, including characterization of alignment vs divergence by behavior (reading, hesitation, scrolling) \citep{Huang2011NoClicks,Huang2012UserSee,Egner2018MouseClickEMAT}, and (2) \emph{endpoint predictability} from unfolding kinematics \citep{Pasqual2014Endpoint}. We turn these into a \emph{frame-free intent channel} and test whether injecting it into an off-the-shelf planner improves control on OSWorld-Verified \citep{OSWorldSite,OSWorldVerified2025}.

\paragraph{Contributions.}
(1) A precise modality specification and encoder for cursor micro-kinematics. (2) Two fusion mechanisms with frozen planners: control-token conditioning and gated cross-attention adapters \citep{Alayrac2022Flamingo,Hu2021LoRA,Li2021PrefixTuning}. (3) A data and training protocol that uses gaze only for offline supervision, never at inference \citep{Latifzadeh2025AdSERP,Leiva2020AttentiveCursor,RecGaze2025}. (4) A reproducible evaluation on OSWorld-Verified with micro-metrics for mis-actions and timing, plus stress tests for delay and device shift \citep{OSWorldSite,OSWorldVerified2025}.

\section{Related Work}
\textbf{Cursor as attention proxy.} Search and browsing studies show cursor positions and movement/hover patterns correlate with gaze and user examination on SERPs; follow-ups quantify when they align vs diverge and define behavior patterns relevant to intent modeling \citep{Huang2011NoClicks,Huang2012UserSee,Egner2018MouseClickEMAT}.

\noindent\textbf{Endpoint prediction from kinematics.} Template matching over early velocity profiles predicts pointing endpoints without requiring target metadata, establishing feasibility for short-window predictors \citep{Pasqual2014Endpoint}.

\noindent\textbf{Motor-control priors.} The Kinematic Theory with the sigma-lognormal model explains sub-movements and lognormal velocity bursts; it motivates using speed/acceleration/jerk and curvature features and optional parametric fits for denoising \citep{Plamondon1995KinematicI,Plamondon2013LognormalHandwriter}.

\noindent\textbf{Modern pointing-time models.} Recent ergonomics work shows trajectory-shape features improve pointing-time prediction beyond Fitts' law, reinforcing that micro-kinematics carry signal \citep{Murata2025PointingTime}.

\noindent\textbf{Adapters and gated cross-attention.} Parameter-efficient adapters and Flamingo-style gated cross-attention layers allow conditioning frozen LMs on auxiliary streams, giving two principled fusion baselines \citep{Hu2021LoRA,Li2021PrefixTuning,Alayrac2022Flamingo}.

\noindent\textbf{Benchmarks for desktop agents.} OSWorld introduced execution-checked desktop tasks and infrastructure; on \emph{July 28, 2025}, OSWorld-Verified addressed community feedback, added scalable evaluation, and publishes verified trajectories/results \citep{OSWorldSite,OSWorldVerified2025}.

\section{Background and Problem Setup}
Let the agent operate at discrete control steps $k$ with a rolling pointer event buffer
\[
\mathcal{B}_k=\{(\Delta x_i,\Delta y_i,\Delta t_i)\}_{i=k-N}^{k}
\]
spanning $T\!\approx\!0.8$--$1.2$\,s. We include light context $c_k$ (focused window/app, modifier keys, wheel deltas, device type).

\textbf{Goal.} From $(\mathcal{B}_k,c_k)$, predict \emph{intent/endpoint variables} that reduce mis-actions and latency without altering the planner’s high-level policy.

\textbf{Outputs per step.}
(i) Endpoint distribution $p(\mathbf{y}\mid \mathcal{B}_k,c_k)=\mathcal{N}(\boldsymbol{\mu}_k,\Sigma_k)$ over screen coords $\mathbf{y}=(x,y)$; (ii) click imminence $q_k=\Pr(\text{click within }\tau)$ for $\tau\in[200,400]$\,ms; (iii) time-to-click $\hat t_k$ (regression). We omit speculative risk classes unless labels are available (see \S\ref{sec:labels}).

\section{Modality and Features}
\textbf{Resampling and normalization.} We resample pointer deltas to 20\,Hz, z-score per session, and normalize by \emph{effective DPI} if known or by screen-diagonal heuristics if not, to counter device heterogeneity (mouse vs touchpad). Public biometrics datasets (e.g., SapiMouse) inform cross-device augmentations \citep{SapiMouse2020}.

\textbf{Kinematic features.} From resampled positions $(x(t),y(t))$: instantaneous speed $v$, acceleration $a$, jerk $j$, signed curvature $\kappa$, heading changes, and stop-and-go indicators. Optionally, fit a short \emph{sigma-lognormal} burst to summarize sub-movements; parameters are used as additional features but not required \citep{Plamondon1995KinematicI,Plamondon2013LognormalHandwriter}.

\textbf{Behavioral windows.} Following search HCI, we tag lightweight behavior windows (reading/hesitation/scrolling) with simple heuristics for auxiliary supervision; these reflect patterns that affect cursor-gaze alignment \citep{Huang2012UserSee}.

\section{Model}
\textbf{Encoder.} A 1D temporal transformer (or Conv-GRU) over $(\Delta x,\Delta y,\Delta t)$ and derived features for the last $T$ seconds. Two heads: \emph{Endpoint head} outputs $\boldsymbol{\mu}\in\mathbb{R}^2$ and a Cholesky-parametrized $\Sigma\succeq 0$. \emph{Timing head} outputs $q\in(0,1)$ and $\hat t\in\mathbb{R}_+$.

\textbf{Fusion with the planner.}
\emph{(F1) Control-tokens}: serialize $(\boldsymbol{\mu},\Sigma,q,\hat t)$ into a fixed schema and append to the planner’s context each step.
\emph{(F2) Gated cross-attention adapter}: insert Flamingo-style gated cross-attention blocks at late layers $\ell\in\{L-6,L-3\}$ of a frozen LM; the cursor encoder outputs act as keys/values, the LM queries the stream, and a tanh-gate scales contributions. Train only the adapter and small projections with parameter-efficient updates (LoRA/Prefix-Tuning) \citep{Alayrac2022Flamingo,Hu2021LoRA,Li2021PrefixTuning}.

\section{Datasets}
We require gaze for \emph{training labels only}, not at inference.
\begin{itemize}\itemsep0pt
  \item \textbf{AdSERP (SIGIR 2025).} In-lab \emph{mouse+eye} on SERPs with HTML, screenshots, per-event logs; includes baselines for cursor$\to$attention decoding. Use fixations as endpoints and AOIs to derive click labels \citep{Latifzadeh2025AdSERP}.
  \item \textbf{Attentive Cursor (2020).} Large-scale cursor traces targeted at attention prediction; useful for representation pretraining \citep{Leiva2020AttentiveCursor}.
  \item \textbf{RecGaze (SIGIR 2025).} Carousel UI interactions with gaze, clicks, cursor; narrower domain for out-of-SERP generalization \citep{RecGaze2025}.
  \item \textbf{Moving-Target-with-Delay (MMSys 2021).} Public datasets from user studies on selecting moving targets under controlled input delay; ideal stress tests for timing \citep{Liu2021MovingDelay}.
  \item \textbf{SapiMouse (2020).} Mouse-dynamics dataset (120 subjects). Not for endpoints, but useful to design device/DPI invariance and augmentations \citep{SapiMouse2020}.
\end{itemize}

\section{Labels and Objectives}
\label{sec:labels}
Let $y^*=(x^*,y^*)$ be the ground-truth \emph{endpoint}. We obtain $y^*$ as: (i) \emph{click endpoint} when a click occurs soon after the window; (ii) \emph{fixation endpoint} (from eye tracking) when no click occurs or for attention pretraining \citep{Huang2011NoClicks,Huang2012UserSee}.\,
Let $t^*$ be \emph{time-to-click} from the current step if a click occurs within $\tau$\,ms; otherwise we drop timing supervision for that step.

\noindent\textbf{Loss.}
\begin{align*}
\mathcal{L} &= \lambda_{e}\,\underbrace{-\log \mathcal{N}\!\left(y^*;\boldsymbol{\mu},\Sigma\right)}_{\text{endpoint NLL}}
\;+\; \lambda_{i}\,\underbrace{\mathrm{BCE}\!\left(q,\mathbb{1}[t^*\le \tau]\right)}_{\text{imminence}}
\;+\; \lambda_{t}\,\underbrace{\mathrm{Huber}\!\left(\hat t,t^*\right)}_{\text{time-to-click}} \\
&\quad+\; \lambda_{\mathrm{cal}}\,\mathrm{ECE}(q,\Sigma)
\;+\; \lambda_{\mathrm{inv}}\,\mathcal{L}_{\mathrm{inv}}.
\end{align*}
$\mathcal{L}_{\mathrm{inv}}$ enforces \emph{device invariance} via augmentation or adversarial confusion across device/DPI domains. We adopt standard calibration penalties (temperature/ECE) to encourage \emph{useful uncertainty} in $q$ and $\Sigma$. The risk classifier is omitted unless target bounding boxes are available (e.g., from AOIs in AdSERP).

\section{Training Protocol}
\textbf{Pretraining.} Stage A: representation pretrain on Attentive Cursor to predict attention heatmaps or fixation likelihood from cursor windows; augment for device variability \citep{Leiva2020AttentiveCursor}. Stage B: supervised endpoint/timing on AdSERP using fixation-derived endpoints and SERP AOIs for click labels \citep{Latifzadeh2025AdSERP}.

\textbf{Adaptation to evaluation domain.} Small supervised finetune using OSWorld instrumented logs (see \S\ref{sec:evaluation}) to match desktop distributions, training only the encoder and fusion modules with parameter-efficient updates (LoRA/Prefix-Tuning) \citep{Hu2021LoRA,Li2021PrefixTuning}.

\textbf{Hyperparameters (recommended ranges).} Window $T=0.8$--$1.2$\,s; resampling 20\,Hz. Encoder: 4--6 layers, $d_{\text{model}}$ 256--512, dropout 0.1. Optimizer: AdamW, lr $2\!\times\!10^{-4}$, cosine decay; batch 256 windows. Fusion: 1--2 gated cross-attn blocks for F2; otherwise control-token schema with 32--64 dims. No screenshots/DOM at training beyond offline label extraction; no pixels at inference.

\section{Evaluation}
\label{sec:evaluation}
\textbf{Benchmark.} \emph{OSWorld-Verified} (upgrade posted 2025-07-28): 369 real tasks, execution-checked scoring, verified trajectories, and scalable evaluation \citep{OSWorldSite,OSWorldVerified2025}.

\textbf{Agent under test.} Keep the baseline planner fixed; compare: \textbf{B0} baseline agent, no kinematics; \textbf{B1} heuristic kinematics (simple velocity/dwell thresholds); \textbf{K-Tok} Cursor-Kinematics with control-tokens (F1); \textbf{K-GXA} Cursor-Kinematics with gated cross-attention adapter (F2) \citep{Alayrac2022Flamingo}.

\textbf{Primary metrics.} \emph{Time-to-first-correct-click (TTFC)} in ms (measured from the step a target becomes selectable to the first correct click); \emph{mis-click rate} (\% of clicks not advancing task state); \emph{retries per task} and \emph{Pass@1}.

\textbf{Micro-metric extraction.} OSWorld-Verified publishes trajectories and code for verified runs; we patch the evaluation harness to log pointer events, click outcomes, and UI activation times to compute TTFC and mis-clicks. We will release the patch \citep{OSWorldVerified2025}.

\textbf{Stress tests.} \emph{Delay robustness}: repeat a 50-task subset with injected input delays (50--150\,ms) and compare TTFC/mis-click deltas; use Moving-Target-with-Delay datasets for complementary, controlled analyses \citep{Liu2021MovingDelay}. \emph{Device shift}: mouse vs touchpad; different pointer speeds. \emph{UI perturbations}: small layout shifts and animation onsets.

\textbf{Statistical protocol.} Three independent seeds, stratified task sampling, paired Wilcoxon tests on per-task metrics; Holm--Bonferroni correction. Report 95\% bootstrap CIs over tasks and Cliff’s $\delta$ effect sizes.

\section{Ablations}
(1) \emph{Window length}: 0.4/0.8/1.2\,s to test lookahead vs latency. (2) \emph{Fusion}: F1 (tokens) vs F2 (gated cross-attention) \citep{Alayrac2022Flamingo}. (3) \emph{Calibration}: with vs without temperature/ECE loss; evaluate sharpness and coverage of $\Sigma$. (4) \emph{Feature sets}: deltas-only vs +kinematics vs +lognormal parameters \citep{Plamondon2013LognormalHandwriter}. (5) \emph{Device invariance}: with vs without $\mathcal{L}_{\mathrm{inv}}$; evaluate on SapiMouse-style augmentations \citep{SapiMouse2020}.

\section{Expected Outcomes (claims to test)}
\textbf{H1:} K-Tok and K-GXA reduce mis-click rate and TTFC vs B0 at fixed Pass@1 on OSWorld-Verified. \textbf{H2:} Gains persist under injected delays (50--150\,ms) and across device settings. \textbf{H3:} Calibrated $\Sigma$ improves the planner’s defer/commit decisions, reducing premature actions.

\section{Limitations and Risk}
\textbf{Domain shift.} Cursor-gaze alignment varies by task (e.g., reading vs navigation); we mitigate via multi-task training and uncertainty-aware fusion, but alignment is not guaranteed \citep{Huang2012UserSee}. \textbf{Label ambiguity.} Fixations are a proxy, not intent itself; therefore we evaluate only on execution-grounded metrics (Pass@k, mis-clicks, TTFC). \textbf{Privacy.} Cursor streams are behavioral data; we log locally within OSWorld VMs, and release only aggregated metrics.

\section{Ethical Considerations}
We avoid eye tracking at inference and reduce reliance on full-frame capture, lowering privacy surface area. We follow dataset licenses and consent terms; for any internal data, obtain IRB/ethics approval and minimize retention.

\section{Implementation Notes}
\textbf{Libraries.} PyTorch; adapters via LoRA; optional Prefix-Tuning for control-token formatting \citep{Hu2021LoRA,Li2021PrefixTuning}. \textbf{Compute.} Encoder and fusion modules run on CPU at 10--20\,Hz; training fits on a single modest GPU. \textbf{Engineering sanity checks.} Gradient clipping; avoid mixed precision in the small encoder to prevent calibration drift; profile latency with wall-clock timers.

\section{Reproducibility Checklist}
Data preprocessing pipelines and splits (released). OSWorld-Verified runner diffs and logging hooks (released) \citep{OSWorldVerified2025}. Configs, seeds, and scripts for all ablations. Pretrained encoder weights and adapter params. Detailed failure taxonomy with examples.

