% CVPR 2026 Paper Template; see https://github.com/cvpr-org/author-kit

\documentclass[10pt,twocolumn,letterpaper]{article}

%%%%%%%%% PAPER TYPE  - PLEASE UPDATE FOR FINAL VERSION
\usepackage{cvpr}              % To produce the CAMERA-READY version
% \usepackage[review]{cvpr}      % To produce the REVIEW version
% \usepackage[pagenumbers]{cvpr} % To force page numbers, e.g. for an arXiv version

% Import additional packages in the preamble file, before hyperref
%% This file contains a number of tweaks that are typically applied to the main document.
%% They are not enabled by default, but can be enabled by uncommenting the relevant lines.

%%
%% Inline annotations; for predefined colors, refer to "dvipsnames" in the xcolor package:
%% https://tinyurl.com/overleaf-colors
%%
\newcommand{\red}[1]{{\color{red}#1}}
\newcommand{\todo}[1]{{\color{red}#1}}
\newcommand{\TODO}[1]{\textbf{\color{red}[TODO: #1]}}
%%
%% disable for camera ready / submission by uncommenting these lines  
%%
% \renewcommand{\TODO}[1]{}
% \renewcommand{\todo}[1]{#1}

%%
%% Common packages used by the proposal
%%
\usepackage{amsmath,amssymb,mathtools}
\usepackage{booktabs}
\usepackage{url}
\usepackage{microtype}
\usepackage{algorithm}
\usepackage{algpseudocode}

%%
%% fine-tune paragraph spacing
%%
% \renewcommand{\paragraph}[1]{\vspace{.5em}\noindent\textbf{#1.}}

%%
%% globally adjusts space between figure and caption
%%
% \setlength{\abovecaptionskip}{.5em}


%%
%% Allows "the use of \paper to refer to the project name"
%% with automatic management of space at the end of the word
%%
% \usepackage{xspace}
% \newcommand{\paper}{ProjectName\xspace}

%%
%% Commonly used math definitions
%%
% \DeclareMathOperator*{\argmin}{arg\,min}
% \DeclareMathOperator*{\argmax}{arg\,max}

%%
%% Tigthen underline
%%
% \usepackage{soul}
% \setuldepth{foobar}


% It is strongly recommended to use hyperref, especially for the review version.
% hyperref with option pagebackref eases the reviewers' job.
% Please disable hyperref *only* if you encounter grave issues, 
% e.g. with the file validation for the camera-ready version.
%
% If you comment hyperref and then uncomment it, you should delete *.aux before re-running LaTeX.
% (Or just hit 'q' on the first LaTeX run, let it finish, and you should be clear).
\definecolor{cvprblue}{rgb}{0.21,0.49,0.74}
\usepackage[pagebackref,breaklinks,colorlinks,allcolors=cvprblue]{hyperref}

%%%%%%%%% PAPER ID  - PLEASE UPDATE
\def\paperID{*****} % *** Enter the Paper ID here
\def\confName{CVPR}
\def\confYear{2026}

%%%%%%%%% TITLE - PLEASE UPDATE
\title{CodeCanvas: Reducing Side-Effect Blindness in Coding Agents}

%%%%%%%%% AUTHORS - PLEASE UPDATE
\author{Ainesh Chatterjee\\
Independent Researcher\\
{\tt\small research@ain3sh.com}
}

\begin{document}
\maketitle
\begin{abstract}
Coding agents break what they can't see. LLM-based code editors often introduce regressions by editing without inspecting the dependency surface: callers, tests, and downstream consumers that must stay consistent. We call this failure mode \emph{side-effect blindness}. The problem worsens when action history falls out of context, so agents forget what they checked or decided and repeat exploration.

We present \textbf{CodeCanvas}, an MCP tool that gives \emph{impact visibility} and preserves reasoning across context windows. CodeCanvas renders a visual codemap of blast radius around the current file or symbol and maintains an Evidence Board that records claims, evidence, and decisions as an action history that survives context compaction. Both artifacts are injected automatically through deterministic hooks, so agents receive impact cues at the moment they choose an edit.

On seven Terminal-Bench 2.0 tasks under a fixed Harbor + Claude Code harness, CodeCanvas solves 4/7 tasks vs 3/7 for both a text-only baseline and a graph-based baseline (LocAgent), with less backtracking (2.14 vs 4.57 average) and lower cost per success (4.56M vs 7.80M tokens). We also propose the \emph{Informed Editing Score} as a process metric for impact-aware editing. Results are descriptive, given single trials per condition due to budget limitations.
\end{abstract}

\vfill\eject

\section{Introduction}
Agents that operate full desktops or browsers still fail on small targets, animated widgets, and transient lag. Humans compensate using \emph{motor micro-signals} visible in ongoing cursor trajectories hundreds of milliseconds before a click. Two strands justify treating cursor dynamics as a first-class input: (1) \emph{cursor $\leftrightarrow$ gaze/attention} correlation on SERPs and browsing, including characterization of alignment vs divergence by behavior (reading, hesitation, scrolling) \citep{Huang2011NoClicks,Huang2012UserSee,Egner2018MouseClickEMAT}, and (2) \emph{endpoint predictability} from unfolding kinematics \citep{Pasqual2014Endpoint}. We turn these into a \emph{frame-free intent channel} and test whether injecting it into an off-the-shelf planner improves control on OSWorld-Verified \citep{OSWorldSite,OSWorldVerified2025}.

\paragraph{Contributions.}
(1) A precise modality specification and encoder for cursor micro-kinematics. (2) Two fusion mechanisms with frozen planners: control-token conditioning and gated cross-attention adapters \citep{Alayrac2022Flamingo,Hu2021LoRA,Li2021PrefixTuning}. (3) A data and training protocol that uses gaze only for offline supervision, never at inference \citep{Latifzadeh2025AdSERP,Leiva2020AttentiveCursor,RecGaze2025}. (4) A reproducible evaluation on OSWorld-Verified with micro-metrics for mis-actions and timing, plus stress tests for delay and device shift \citep{OSWorldSite,OSWorldVerified2025}.

\section{Related Work}
\textbf{Cursor as attention proxy.} Search and browsing studies show cursor positions and movement/hover patterns correlate with gaze and user examination on SERPs; follow-ups quantify when they align vs diverge and define behavior patterns relevant to intent modeling \citep{Huang2011NoClicks,Huang2012UserSee,Egner2018MouseClickEMAT}.

\noindent\textbf{Endpoint prediction from kinematics.} Template matching over early velocity profiles predicts pointing endpoints without requiring target metadata, establishing feasibility for short-window predictors \citep{Pasqual2014Endpoint}.

\noindent\textbf{Motor-control priors.} The Kinematic Theory with the sigma-lognormal model explains sub-movements and lognormal velocity bursts; it motivates using speed/acceleration/jerk and curvature features and optional parametric fits for denoising \citep{Plamondon1995KinematicI,Plamondon2013LognormalHandwriter}.

\noindent\textbf{Modern pointing-time models.} Recent ergonomics work shows trajectory-shape features improve pointing-time prediction beyond Fitts' law, reinforcing that micro-kinematics carry signal \citep{Murata2025PointingTime}.

\noindent\textbf{Adapters and gated cross-attention.} Parameter-efficient adapters and Flamingo-style gated cross-attention layers allow conditioning frozen LMs on auxiliary streams, giving two principled fusion baselines \citep{Hu2021LoRA,Li2021PrefixTuning,Alayrac2022Flamingo}.

\noindent\textbf{Benchmarks for desktop agents.} OSWorld introduced execution-checked desktop tasks and infrastructure; on \emph{July 28, 2025}, OSWorld-Verified addressed community feedback, added scalable evaluation, and publishes verified trajectories/results \citep{OSWorldSite,OSWorldVerified2025}.

\section{Background and Problem Setup}
Let the agent operate at discrete control steps $k$ with a rolling pointer event buffer
\[
\mathcal{B}_k=\{(\Delta x_i,\Delta y_i,\Delta t_i)\}_{i=k-N}^{k}
\]
spanning $T\!\approx\!0.8$--$1.2$\,s. We include light context $c_k$ (focused window/app, modifier keys, wheel deltas, device type).

\textbf{Goal.} From $(\mathcal{B}_k,c_k)$, predict \emph{intent/endpoint variables} that reduce mis-actions and latency without altering the planner’s high-level policy.

\textbf{Outputs per step.}
(i) Endpoint distribution $p(\mathbf{y}\mid \mathcal{B}_k,c_k)=\mathcal{N}(\boldsymbol{\mu}_k,\Sigma_k)$ over screen coords $\mathbf{y}=(x,y)$; (ii) click imminence $q_k=\Pr(\text{click within }\tau)$ for $\tau\in[200,400]$\,ms; (iii) time-to-click $\hat t_k$ (regression). We omit speculative risk classes unless labels are available (see \S\ref{sec:labels}).

\section{Modality and Features}
\textbf{Resampling and normalization.} We resample pointer deltas to 20\,Hz, z-score per session, and normalize by \emph{effective DPI} if known or by screen-diagonal heuristics if not, to counter device heterogeneity (mouse vs touchpad). Public biometrics datasets (e.g., SapiMouse) inform cross-device augmentations \citep{SapiMouse2020}.

\textbf{Kinematic features.} From resampled positions $(x(t),y(t))$: instantaneous speed $v$, acceleration $a$, jerk $j$, signed curvature $\kappa$, heading changes, and stop-and-go indicators. Optionally, fit a short \emph{sigma-lognormal} burst to summarize sub-movements; parameters are used as additional features but not required \citep{Plamondon1995KinematicI,Plamondon2013LognormalHandwriter}.

\textbf{Behavioral windows.} Following search HCI, we tag lightweight behavior windows (reading/hesitation/scrolling) with simple heuristics for auxiliary supervision; these reflect patterns that affect cursor-gaze alignment \citep{Huang2012UserSee}.

\section{Model}
\textbf{Encoder.} A 1D temporal transformer (or Conv-GRU) over $(\Delta x,\Delta y,\Delta t)$ and derived features for the last $T$ seconds. Two heads: \emph{Endpoint head} outputs $\boldsymbol{\mu}\in\mathbb{R}^2$ and a Cholesky-parametrized $\Sigma\succeq 0$. \emph{Timing head} outputs $q\in(0,1)$ and $\hat t\in\mathbb{R}_+$.

\textbf{Fusion with the planner.}
\emph{(F1) Control-tokens}: serialize $(\boldsymbol{\mu},\Sigma,q,\hat t)$ into a fixed schema and append to the planner’s context each step.
\emph{(F2) Gated cross-attention adapter}: insert Flamingo-style gated cross-attention blocks at late layers $\ell\in\{L-6,L-3\}$ of a frozen LM; the cursor encoder outputs act as keys/values, the LM queries the stream, and a tanh-gate scales contributions. Train only the adapter and small projections with parameter-efficient updates (LoRA/Prefix-Tuning) \citep{Alayrac2022Flamingo,Hu2021LoRA,Li2021PrefixTuning}.

\section{Datasets}
We require gaze for \emph{training labels only}, not at inference.
\begin{itemize}\itemsep0pt
  \item \textbf{AdSERP (SIGIR 2025).} In-lab \emph{mouse+eye} on SERPs with HTML, screenshots, per-event logs; includes baselines for cursor$\to$attention decoding. Use fixations as endpoints and AOIs to derive click labels \citep{Latifzadeh2025AdSERP}.
  \item \textbf{Attentive Cursor (2020).} Large-scale cursor traces targeted at attention prediction; useful for representation pretraining \citep{Leiva2020AttentiveCursor}.
  \item \textbf{RecGaze (SIGIR 2025).} Carousel UI interactions with gaze, clicks, cursor; narrower domain for out-of-SERP generalization \citep{RecGaze2025}.
  \item \textbf{Moving-Target-with-Delay (MMSys 2021).} Public datasets from user studies on selecting moving targets under controlled input delay; ideal stress tests for timing \citep{Liu2021MovingDelay}.
  \item \textbf{SapiMouse (2020).} Mouse-dynamics dataset (120 subjects). Not for endpoints, but useful to design device/DPI invariance and augmentations \citep{SapiMouse2020}.
\end{itemize}

\section{Labels and Objectives}
\label{sec:labels}
Let $y^*=(x^*,y^*)$ be the ground-truth \emph{endpoint}. We obtain $y^*$ as: (i) \emph{click endpoint} when a click occurs soon after the window; (ii) \emph{fixation endpoint} (from eye tracking) when no click occurs or for attention pretraining \citep{Huang2011NoClicks,Huang2012UserSee}.\,
Let $t^*$ be \emph{time-to-click} from the current step if a click occurs within $\tau$\,ms; otherwise we drop timing supervision for that step.

\noindent\textbf{Loss.}
\begin{align*}
\mathcal{L} &= \lambda_{e}\,\underbrace{-\log \mathcal{N}\!\left(y^*;\boldsymbol{\mu},\Sigma\right)}_{\text{endpoint NLL}}
\;+\; \lambda_{i}\,\underbrace{\mathrm{BCE}\!\left(q,\mathbb{1}[t^*\le \tau]\right)}_{\text{imminence}}
\;+\; \lambda_{t}\,\underbrace{\mathrm{Huber}\!\left(\hat t,t^*\right)}_{\text{time-to-click}} \\
&\quad+\; \lambda_{\mathrm{cal}}\,\mathrm{ECE}(q,\Sigma)
\;+\; \lambda_{\mathrm{inv}}\,\mathcal{L}_{\mathrm{inv}}.
\end{align*}
$\mathcal{L}_{\mathrm{inv}}$ enforces \emph{device invariance} via augmentation or adversarial confusion across device/DPI domains. We adopt standard calibration penalties (temperature/ECE) to encourage \emph{useful uncertainty} in $q$ and $\Sigma$. The risk classifier is omitted unless target bounding boxes are available (e.g., from AOIs in AdSERP).

\section{Training Protocol}
\textbf{Pretraining.} Stage A: representation pretrain on Attentive Cursor to predict attention heatmaps or fixation likelihood from cursor windows; augment for device variability \citep{Leiva2020AttentiveCursor}. Stage B: supervised endpoint/timing on AdSERP using fixation-derived endpoints and SERP AOIs for click labels \citep{Latifzadeh2025AdSERP}.

\textbf{Adaptation to evaluation domain.} Small supervised finetune using OSWorld instrumented logs (see \S\ref{sec:evaluation}) to match desktop distributions, training only the encoder and fusion modules with parameter-efficient updates (LoRA/Prefix-Tuning) \citep{Hu2021LoRA,Li2021PrefixTuning}.

\textbf{Hyperparameters (recommended ranges).} Window $T=0.8$--$1.2$\,s; resampling 20\,Hz. Encoder: 4--6 layers, $d_{\text{model}}$ 256--512, dropout 0.1. Optimizer: AdamW, lr $2\!\times\!10^{-4}$, cosine decay; batch 256 windows. Fusion: 1--2 gated cross-attn blocks for F2; otherwise control-token schema with 32--64 dims. No screenshots/DOM at training beyond offline label extraction; no pixels at inference.

\section{Evaluation}
\label{sec:evaluation}
\textbf{Benchmark.} \emph{OSWorld-Verified} (upgrade posted 2025-07-28): 369 real tasks, execution-checked scoring, verified trajectories, and scalable evaluation \citep{OSWorldSite,OSWorldVerified2025}.

\textbf{Agent under test.} Keep the baseline planner fixed; compare: \textbf{B0} baseline agent, no kinematics; \textbf{B1} heuristic kinematics (simple velocity/dwell thresholds); \textbf{K-Tok} Cursor-Kinematics with control-tokens (F1); \textbf{K-GXA} Cursor-Kinematics with gated cross-attention adapter (F2) \citep{Alayrac2022Flamingo}.

\textbf{Primary metrics.} \emph{Time-to-first-correct-click (TTFC)} in ms (measured from the step a target becomes selectable to the first correct click); \emph{mis-click rate} (\% of clicks not advancing task state); \emph{retries per task} and \emph{Pass@1}.

\textbf{Micro-metric extraction.} OSWorld-Verified publishes trajectories and code for verified runs; we patch the evaluation harness to log pointer events, click outcomes, and UI activation times to compute TTFC and mis-clicks. We will release the patch \citep{OSWorldVerified2025}.

\textbf{Stress tests.} \emph{Delay robustness}: repeat a 50-task subset with injected input delays (50--150\,ms) and compare TTFC/mis-click deltas; use Moving-Target-with-Delay datasets for complementary, controlled analyses \citep{Liu2021MovingDelay}. \emph{Device shift}: mouse vs touchpad; different pointer speeds. \emph{UI perturbations}: small layout shifts and animation onsets.

\textbf{Statistical protocol.} Three independent seeds, stratified task sampling, paired Wilcoxon tests on per-task metrics; Holm--Bonferroni correction. Report 95\% bootstrap CIs over tasks and Cliff’s $\delta$ effect sizes.

\section{Ablations}
(1) \emph{Window length}: 0.4/0.8/1.2\,s to test lookahead vs latency. (2) \emph{Fusion}: F1 (tokens) vs F2 (gated cross-attention) \citep{Alayrac2022Flamingo}. (3) \emph{Calibration}: with vs without temperature/ECE loss; evaluate sharpness and coverage of $\Sigma$. (4) \emph{Feature sets}: deltas-only vs +kinematics vs +lognormal parameters \citep{Plamondon2013LognormalHandwriter}. (5) \emph{Device invariance}: with vs without $\mathcal{L}_{\mathrm{inv}}$; evaluate on SapiMouse-style augmentations \citep{SapiMouse2020}.

\section{Expected Outcomes (claims to test)}
\textbf{H1:} K-Tok and K-GXA reduce mis-click rate and TTFC vs B0 at fixed Pass@1 on OSWorld-Verified. \textbf{H2:} Gains persist under injected delays (50--150\,ms) and across device settings. \textbf{H3:} Calibrated $\Sigma$ improves the planner’s defer/commit decisions, reducing premature actions.

\section{Limitations and Risk}
\textbf{Domain shift.} Cursor-gaze alignment varies by task (e.g., reading vs navigation); we mitigate via multi-task training and uncertainty-aware fusion, but alignment is not guaranteed \citep{Huang2012UserSee}. \textbf{Label ambiguity.} Fixations are a proxy, not intent itself; therefore we evaluate only on execution-grounded metrics (Pass@k, mis-clicks, TTFC). \textbf{Privacy.} Cursor streams are behavioral data; we log locally within OSWorld VMs, and release only aggregated metrics.

\section{Ethical Considerations}
We avoid eye tracking at inference and reduce reliance on full-frame capture, lowering privacy surface area. We follow dataset licenses and consent terms; for any internal data, obtain IRB/ethics approval and minimize retention.

\section{Implementation Notes}
\textbf{Libraries.} PyTorch; adapters via LoRA; optional Prefix-Tuning for control-token formatting \citep{Hu2021LoRA,Li2021PrefixTuning}. \textbf{Compute.} Encoder and fusion modules run on CPU at 10--20\,Hz; training fits on a single modest GPU. \textbf{Engineering sanity checks.} Gradient clipping; avoid mixed precision in the small encoder to prevent calibration drift; profile latency with wall-clock timers.

\section{Reproducibility Checklist}
Data preprocessing pipelines and splits (released). OSWorld-Verified runner diffs and logging hooks (released) \citep{OSWorldVerified2025}. Configs, seeds, and scripts for all ablations. Pretrained encoder weights and adapter params. Detailed failure taxonomy with examples.


\section{Related Work}
\label{sec:related}

\subsection{Terminal-Executed Agent Evaluation}
Terminal-Bench frames agent competence as \emph{end-to-end terminal execution}: agents must navigate containerized environments, run commands, inspect artifacts, and complete multi-step workflows with verifiable outcomes~\cite{terminalbench2025}. Terminal-Bench~2.0 and Harbor emphasize stronger verification and reproducible orchestration~\cite{terminalbench2harbor2025}. Harbor additionally standardizes trajectory logging via ATIF, enabling consistent process-level analyses (tokens, tool calls, observations) across agents and harnesses~\cite{harborATIF2025}.

\subsection{Harness Design and Agent-Computer Interfaces}
Agent performance depends strongly on the agent-computer interface (ACI): action primitives, observation formatting, and feedback channels shape what strategies are feasible and how reliably they execute~\cite{sweagent2024}. Systems such as OpenHands further emphasize deterministic harnesses and sandboxed execution to support reproducible evaluation~\cite{openhands2024}. CodeCanvas targets a narrow, experimentally controlled axis within this space: holding the harness constant (Harbor + Claude Code), we vary only the \emph{observation channel} delivered through MCP tools and deterministic hooks~\cite{claudeCodeHooks2025}.

\subsection{Repository Representations for Search and Localization}
Repository-scale tasks stress context selection: relevant information is distributed across files, interfaces, and dependency structure. Retrieval-centric work focuses on efficient \emph{textual} context selection~\cite{repocoder2023,repoformer2024,repofuse2024}. Complementarily, graph-based representations make structure explicit. RepoGraph provides repository-level graphs to guide navigation~\cite{repograph2024}, and LocAgent constructs heterogeneous dependency graphs and applies multi-hop traversal for localization~\cite{locagent2025}. These systems motivate our graph-based baseline: structural signals can reduce ambiguity and improve search, but they may also impose a query burden and fragment understanding across tool calls.

\subsection{Dependency-Aware Grounding for Correct Edits}
Even with correct localization, agents often fail due to project-context mismatch: invalid API usage, broken dependencies, or refactors that neglect callers and tests. CoCoGen combines compiler feedback with iterative context refinement to improve project-context-dependent code generation~\cite{cocogen2024}. MARIN constrains decoding to valid project APIs by mining hierarchical dependencies~\cite{marin2025}. These approaches share a common principle: \emph{dependency structure} is essential for correctness in realistic codebases. CodeCanvas is complementary: rather than constraining generation, it surfaces dependency impact as an observation \emph{before} edits are made.

\subsection{External Memory and Visual Reasoning Artifacts}
Long-horizon interactive tasks exceed context windows and require maintaining hypotheses, evidence, and decisions across iterations. MemGPT proposes explicit external memory with control flow for long-context settings~\cite{memgpt2023}. In parallel, multimodal work shows that visual intermediate artifacts can improve reasoning when the underlying structure is spatial or graph-like~\cite{whiteboard2024,visualsketchpad2024}. Closest to our setting, visualization can also support monitoring and steering of code-driven workflows~\cite{waitgpt2024}. CodeCanvas combines both ideas: it externalizes reasoning state via an Evidence Board and makes dependency structure consumable via compact codemaps.

\subsection{Positioning}
Prior work establishes the importance of harness/ACI design~\cite{sweagent2024,terminalbench2harbor2025}, structured representations for navigation~\cite{locagent2025,repograph2024}, dependency-aware grounding for correctness~\cite{cocogen2024,marin2025}, and visual artifacts for structured reasoning~\cite{whiteboard2024,visualsketchpad2024,waitgpt2024}. However, there is limited controlled evidence in a \emph{terminal-executed}, fixed-harness setting isolating how \emph{observation modality} (text-only vs.\ graph-derived signals vs.\ visual codemaps) and \emph{persistence via deterministic hooks} affect stability and efficiency. Our work addresses this gap by evaluating these design choices on Terminal-Bench~2.0 under Harbor with Claude Code.

\section{Method: CodeCanvas}
\label{sec:method}

\subsection{Design Principles}
\label{sec:design-goals}

CodeCanvas is designed around three principles:
\begin{enumerate}
    \item \textbf{Observable Blast Radius.} Before editing, the agent should see what code depends on the focus region: callers, tests, downstream consumers.
    \item \textbf{Persistent Reasoning State.} Claims, evidence, and decisions should survive context compaction, enabling coherent multi-step reasoning.
    \item \textbf{Reduced Query Burden.} Deterministic hooks inject context automatically when CodeCanvas is active; the agent need not remember to invoke analysis tools.
\end{enumerate}

\subsection{Architecture Overview}
\label{sec:architecture}

CodeCanvas is packaged as an MCP server exposing a single tool, \texttt{canvas}. It supports (i) session initialization and repository parsing (\texttt{init}), (ii) blast-radius slicing for a file or symbol (\texttt{impact}), and (iii) persistent state updates for claims and decisions (\texttt{claim}, \texttt{decide}) with lightweight progress tracking (\texttt{mark}, \texttt{skip}, \texttt{status}, \texttt{read}).

Internally, CodeCanvas uses LSP-first parsing (e.g., pylsp, tsserver, clangd) with tree-sitter fallback for unsupported or failing language servers. All state and generated artifacts are persisted under \texttt{.codecanvas/} in the workspace.

\begin{figure*}[t]
\centering
\includegraphics[width=0.98\textwidth]{fig/architecture_example.png}
\caption{Repository architecture view from \texttt{sanitize-git-repo}, grouping modules into layers and surfacing cross-layer dependencies for navigation and triage.}
\label{fig:architecture-example}
\end{figure*}

\subsection{Blast Radius Visualization}
\label{sec:blast-radius}

We operationalize \emph{blast radius} as a bounded slice of the dependency surface obtained via LSP queries: for symbols in the focus region, we collect reference locations and call-hierarchy neighbors and map them to file paths. CodeCanvas renders this slice and its relations as a compact \emph{codemap} artifact.

When the agent reads a file, the \texttt{PostToolUse:Read} hook triggers impact analysis:
\begin{enumerate}
    \item Extract symbols from the focus region using LSP \texttt{documentSymbol}
    \item Query references and dependents via LSP \texttt{references} and \texttt{callHierarchy}
    \item Render a codemap showing dependency structure
    \item Annotate with caller counts, test file indicators, and modification hints
\end{enumerate}

\begin{figure}[t]
\centering
\includegraphics[width=0.98\linewidth]{fig/impact_example.png}
\caption{Example blast-radius codemap for \texttt{pyknotid/make/torus.py} (from \texttt{build-cython-ext}), surfacing upstream and downstream modules affected by edits to the target file.}
\label{fig:codemap-example}
\end{figure}

\subsection{Evidence Board}
\label{sec:evidence-board}

The Evidence Board provides three primitives for externalizing reasoning:
\begin{description}
    \item[Claim] A hypothesis about the codebase or task (e.g., ``The bug is in the parser module'')
    \item[Evidence] An observation supporting or refuting a claim (e.g., ``Found stack trace pointing to line 42'')
    \item[Decision] A committed action with rationale (e.g., ``Will refactor \texttt{parse\_input} to handle edge case'')
\end{description}

The board is rendered as a compact artifact that survives compaction and is re-injected via hooks.

\begin{figure}[t]
\centering
\includegraphics[width=0.98\linewidth]{fig/evidence_board.png}
\caption{Evidence Board interface from the \texttt{fix-code-vulnerability} task. Left: Claims summarize suspected vulnerabilities (e.g., CWE-502, CWE-22). Center: Evidence thumbnails link to impact analyses. Right: Decisions record planned and executed actions and verification status.}
\label{fig:evidence-board}
\end{figure}

\subsection{Hook Integration}
\label{sec:hooks}

CodeCanvas uses Claude Code hooks for deterministic augmentation:
\begin{itemize}
    \item \textbf{SessionStart}: Auto-initialize canvas state and inject Evidence Board summary
    \item \textbf{PostToolUse:Read}: Trigger impact analysis for any file read, injecting codemap into observation
\end{itemize}

In our implementation, SessionStart initialization is gated by a lightweight repository check and a minimum threshold on recognized code files to avoid overhead in non-code directories; when the gate fails to trigger, CodeCanvas does not produce visual artifacts for that task.

\section{Experimental Setup}
\label{sec:experiments}

\subsection{Benchmark: Terminal-Bench 2.0}
\label{sec:benchmark}

We evaluate on Terminal-Bench~2.0~\cite{terminalbench2harbor2025}, which emphasizes end-to-end terminal execution with verifiable outcomes. We select seven tasks spanning diverse domains and complexity:

\begin{table}[h]
\centering
\scriptsize
\begin{tabular}{@{}p{2.4cm}llc@{}}
\toprule
\textbf{Task} & \textbf{Domain} & \textbf{Language} & \textbf{Difficulty} \\
\midrule
sanitize-git-repo & Security & Shell/Git & Medium \\
build-cython-ext & Compilation & Python/C & Medium \\
custom-memory-heap-crash & Debugging & C++ & Medium \\
db-wal-recovery & Database & SQL/Python & Medium \\
modernize-scientific-stack & Migration & Python & Medium \\
rstan-to-pystan & Translation & R/Python & Medium \\
fix-code-vulnerability & Security & Python & Hard \\
\bottomrule
\end{tabular}
\caption{Terminal-Bench 2.0 tasks used in evaluation.}
\label{tab:tasks}
\end{table}

Tasks are executed in Harbor-managed containers with build fingerprinting for reproducibility~\cite{terminalbench2harbor2025}.

~

~

~

~

~

~

\subsection{Agent Harness}
\label{sec:harness}

All conditions use Claude Code via a custom \texttt{ClaudeCodeMCP} wrapper, executed under Harbor with multi-profile support (\texttt{-C} flags). Trajectories are logged in ATIF format for post-hoc analysis~\cite{harborATIF2025}.

\subsection{Experimental Conditions}
\label{sec:conditions}

We compare three observation modalities while holding the harness constant:

\begin{table}[h]
\centering
\scriptsize
\begin{tabular}{@{}p{1.6cm}p{2.2cm}p{2.9cm}@{}}
\toprule
\textbf{Condition} & \textbf{MCP Tools} & \textbf{Hooks} \\
\midrule
Text-Only & None & None \\
LocAgent & LocAgent~(5 tools) & None \\
CodeCanvas & canvas~(8 actions) & SessionStart,~PostToolUse \\
\bottomrule
\end{tabular}
\caption{Experimental conditions. LocAgent~\cite{locagent2025} is used as the graph-based MCP baseline.}
\label{tab:conditions}
\end{table}

\noindent\textbf{CodeCanvas activation.}
SessionStart initialization is gated by (i) detection of a code repository marker and (ii) a minimum number of recognized code files. If initialization does not trigger, CodeCanvas produces no state artifacts for that run and CodeCanvas-specific process metrics are undefined. In our run set, CodeCanvas state was recorded on 2/7 tasks (and IES is computed only on that subset).

\noindent\textbf{LocAgent baseline packaging.}
LocAgent~\cite{locagent2025} is originally presented as a full agent framework. To isolate the effect of observation modality, we repackage its dependency-graph construction and retrieval primitives as an MCP server and run them under Claude Code, removing baseline-specific orchestration from the comparison.
This repackaging may degrade LocAgent relative to its original implementation; we therefore interpret results as LocAgent primitives under Claude Code rather than LocAgent as a complete system.

\paragraph{Run batches (mixed reporting).}
The 21 trajectories analyzed in this paper derive from two Terminal-Bench run batches sharing identical tasks, model, and Harbor + Claude Code harness. Text-Only and CodeCanvas trajectories are drawn from an earlier batch, while LocAgent trajectories are drawn from a later batch after baseline packaging changes. We report this mixed-batch aggregate to preserve the corrected LocAgent baseline and treat all comparisons as descriptive (single trial per task and condition).

~

~

~

\paragraph{Hypotheses (exploratory).}
We structure our analysis around two design hypotheses, framed as qualitative expectations rather than statistical claims:
\begin{itemize}
    \item \textbf{H1 (Impact visibility reduces revision churn).} When the dependency surface is made explicit at edit time (codemaps), the agent performs fewer revision cycles, reflected in lower backtrack and loop counts.
    \item \textbf{H2 (Persistent state reduces redundant work).} When claims/evidence/decisions persist across context compaction and are re-injected via hooks, the agent repeats fewer identical actions, reflected in lower loop counts and fewer tool calls.
\end{itemize}

\subsection{Metrics}
\label{sec:metrics}

\paragraph{Primary Outcomes.}
\begin{itemize}
    \item \textbf{Pass@1}: Task success rate (one run per task and condition)
\end{itemize}

\paragraph{Process Metrics.}
\begin{itemize}
    \item Token usage (input + output)
    \item Step count (tool invocations)
    \item Tool invocation density (tools per step)
    \item \textbf{Backtrack count}: number of edit actions that revisit a file edited within the previous three edit actions (detected over \texttt{Edit}/\texttt{MultiEdit}/\texttt{Create} tool calls)
    \item \textbf{Loop count}: number of tool calls whose (tool name, arguments) repeats within the previous five tool calls
    \item \textbf{Grep-before-edit}: whether any search call (\texttt{Grep}/\texttt{Glob} or LocAgent \texttt{search\_code}) occurs before the first \texttt{Edit} or \texttt{MultiEdit}
\end{itemize}

\paragraph{CodeCanvas-Specific: Informed Editing Score (IES).}
We propose IES as a process diagnostic that summarizes (i) how much editing stays within the analyzed blast radius, (ii) whether failing tests were anticipated by impact analysis, and (iii) evidence of deliberation via recorded claims and decisions. The full definition is given in Appendix~\ref{app:ies}. IES is computed only for runs where CodeCanvas state was recorded (n=2) and should be interpreted as preliminary.

\subsection{Analysis Protocol}
\label{sec:analysis-protocol}

We employ a two-layer analysis framework:
\begin{description}
    \item[Layer 1 (Deterministic):] Free metrics extracted directly from ATIF trajectories: step counts, token usage, tool patterns, success/failure outcomes.
    \item[Layer 2 (Automated):] GPT-5.2 semantic analysis (\$0.05/trajectory) for failure attribution, labeling failures into the five modes in Section~\ref{sec:failure-modes}. These labels are treated as automated heuristics rather than ground truth.
\end{description}

\section{Results}
\label{sec:results}

We analyze 21 trajectories (7 tasks $\times$ 3 conditions; one run per task and condition). As described in Section~\ref{sec:conditions}, these trajectories come from two run batches (Text-Only and CodeCanvas from an earlier batch; LocAgent from a later batch). Table~\ref{tab:aggregate-results} summarizes aggregate performance; Figure~\ref{fig:per-task-heatmap} visualizes per-task outcomes and token usage. In this run set, the CodeCanvas condition solves 4/7 tasks (57.1\%), while Text-Only and LocAgent each solve 3/7 (42.9\%). With $n=7$, single trials per condition, and mixed-batch provenance, we treat differences as descriptive. Because CodeCanvas activation is gated (Section~\ref{sec:conditions}), CodeCanvas state was recorded on only 2/7 tasks; CodeCanvas-specific process metrics (IES) are reported only on that subset.

\subsection{Primary Outcomes}
\label{sec:primary-outcomes}

Table~\ref{tab:aggregate-results} reports Pass@1 and average resource usage across tasks. Figure~\ref{fig:tokens-vs-backtrack} summarizes trajectory-level efficiency as tokens vs.\ backtracking.

\begin{table}[t]
\centering
\scriptsize
\begin{tabular}{@{}lcccc@{}}
\toprule
\textbf{Condition} & \textbf{Pass@1} & \textbf{Tokens (M)} & \textbf{Steps} & \textbf{Time (m)} \\
\midrule
Text-Only & 3/7 (42.9\%) & 3.34 & 86.1 & 10.0 \\
LocAgent & 3/7 (42.9\%) & 2.48 & 76.3 & 8.5 \\
CodeCanvas & 4/7 (57.1\%) & 2.61 & 72.7 & 7.1 \\
\bottomrule
\end{tabular}
\caption{Aggregate results across all tasks (7 tasks; one run per task and condition). Tokens are input+output; comparisons are descriptive given mixed-batch provenance.}
\label{tab:aggregate-results}
\end{table}

\begin{figure}[t]
\centering
\includegraphics[width=0.98\linewidth]{fig/tokens_vs_backtrack.png}
\caption{Trajectory-level efficiency (21 trajectories): tokens vs.\ backtracking. Filled markers indicate success; x-axis is log-scaled. The y-axis is capped at 10; the single outlier (22 backtracks) is annotated.}
\label{fig:tokens-vs-backtrack}
\end{figure}

\subsection{Per-Task Breakdown}
\label{sec:per-task}

\begin{figure}[t]
\centering
\includegraphics[width=0.98\linewidth]{fig/per_task_heatmap.png}
\caption{Per-task outcomes and token usage. Each cell shows success (\,\checkmark\,/\,$\times$\,) and total tokens (M); background color indicates token usage (log scale). Task labels are abbreviated for readability.}
\label{fig:per-task-heatmap}
\end{figure}

\noindent\textbf{Where modalities differ.} Outcomes differ on four tasks: only CodeCanvas succeeds on \emph{rstan-to-pystan}; only Text-Only fails on \emph{custom-memory-heap-crash}; only LocAgent fails on \emph{build-cython-ext}; and only CodeCanvas fails on \emph{fix-code-vulnerability}. All conditions succeed on \emph{modernize-scientific-stack} and fail on \emph{sanitize-git-repo} and \emph{db-wal-recovery}.

\paragraph{Informed Editing Score (IES).}
We report IES as a preliminary process diagnostic on the two CodeCanvas trajectories with recorded state (Appendix~\ref{app:ies}).

~

~

~

~

\section{Analysis}
\label{sec:analysis}

We analyze ATIF trajectories to characterize how observation modality shapes behavior beyond binary success. Layer~2 semantic analysis provides failure root-cause labels, while Layer~1 metrics quantify search, editing, and verification patterns.

\subsection{Tool Invocation Patterns}
\label{sec:tool-use}

Pre-edit search behavior varies across conditions. The grep-before-edit rate is 42.9\% for Text-Only and CodeCanvas and 57.1\% for LocAgent, suggesting the graph baseline relies more heavily on search before editing. Search frequency alone does not explain outcomes: the CodeCanvas condition solves 4/7 tasks and exhibits lower backtracking despite matching Text-Only on grep-before-edit.

\noindent\textbf{Activation coverage.} CodeCanvas state was recorded on 2/7 tasks due to a SessionStart heuristic requiring at least five recognized code files. Those are also the only tasks with CodeCanvas MCP tool calls (\texttt{sanitize-git-repo} and \texttt{fix-code-vulnerability}), and both fail; CodeCanvas-specific artifacts therefore mostly characterize failure cases here. Tool-use patterns remain measurable across all trajectories: Figure~\ref{fig:tool-mix} shows the CodeCanvas condition uses fewer total tool calls than both baselines.

\begin{figure}[t]
\centering
\includegraphics[width=0.98\linewidth]{fig/tool_mix.png}
\caption{Tool-call mix across conditions (sum over 7 tasks). Numbers above bars denote total tool calls.}
\label{fig:tool-mix}
\end{figure}

~

~

~

~

~

~

\subsection{Failure Mode Taxonomy}
\label{sec:failure-modes}

We categorize each failed trajectory into one of five modes:
\textbf{Poor Execution} (plausible approach, missed details);
\textbf{Tool Misuse} (correct tools used destructively);
\textbf{Incomplete Exploration} (failed to find relevant code);
\textbf{Misunderstood Task} (wrong objective);
\textbf{Premature Termination} (ran out of budget/time).

\begin{table}[t]
\centering
\small
\begin{tabular}{@{}p{2.7cm}ccc@{}}
\toprule
\textbf{Failure Mode} & \textbf{Text} & \textbf{LocAgent} & \textbf{CodeCanvas} \\
\midrule
Poor Execution & 50.0\% & 50.0\% & 33.3\% \\
Tool Misuse & 25.0\% & 25.0\% & 66.7\% \\
Incomplete Exploration & 0.0\% & 0.0\% & 0.0\% \\
Misunderstood Task & 0.0\% & 25.0\% & 0.0\% \\
Premature Termination & 25.0\% & 0.0\% & 0.0\% \\
\bottomrule
\end{tabular}
\caption{Failure mode distribution (\% of failures) from Layer~2 analysis.}
\label{tab:failure-modes}
\end{table}

\noindent Denominators are the number of failed runs per condition (Text-Only: 4, LocAgent: 4, CodeCanvas: 3).

\noindent\textbf{Common failure points.} Across all conditions, \emph{db-wal-recovery} failures are attributable to procedurally unsafe interaction with the live database (opening a WAL-mode database before preserving the WAL). For CodeCanvas on \emph{fix-code-vulnerability}, the core code changes were close to correct but the run failed due to a strict report-schema mismatch, illustrating that impact awareness does not substitute for contract checking.

~

~

~

~

~

~

~

~

\subsection{Efficiency vs.\ Effectiveness}
\label{sec:efficiency}

\begin{table}[t]
\centering
\small
\begin{tabular}{@{}lccc@{}}
\toprule
\textbf{Metric} & \textbf{Text} & \textbf{LocAgent} & \textbf{CodeCanvas} \\
\midrule
Avg Tokens (M) & 3.34 & \textbf{2.48} & 2.61 \\
Avg Backtrack & 4.57 & 2.57 & \textbf{2.14} \\
Avg Loop & 0.86 & 0.29 & \textbf{0.14} \\
Tokens / Success (M) & 7.80 & 5.79 & \textbf{4.56} \\
Time / Success (m) & 23.3 & 19.8 & \textbf{12.4} \\
Steps / Success & 201.0 & 178.0 & \textbf{127.3} \\
\bottomrule
\end{tabular}
\caption{Efficiency metrics (lower is better). Tokens/time/steps per success include failed trajectories in the numerator.}
\label{tab:efficiency}
\end{table}

\noindent\textbf{Efficiency and effectiveness diverge across baselines.} LocAgent uses the fewest tokens per run on average (2.48M) but does not improve Pass@1 relative to Text-Only (both 42.9\%). In this run set, normalizing by success, the CodeCanvas condition has the lowest cost per solved task (4.56M tokens and 12.4 minutes per success vs.\ 5.79M/19.8m for LocAgent and 7.80M/23.3m for Text-Only). Figure~\ref{fig:tokens-vs-backtrack} shows this pattern at the trajectory level: the CodeCanvas condition achieves comparable or lower backtracking at moderate token budgets, while Text-Only exhibits higher-cost failures and higher backtracking.

\subsection{Scope of Representation Benefits}
\label{sec:where-helps}

In this run set, the CodeCanvas condition uniquely succeeds on \emph{rstan-to-pystan} and shows lower backtracking and looping than both baselines. Because CodeCanvas state is recorded on only two tasks in this study, we cannot directly connect aggregate outcomes to codemap-guided impact analysis; instead, we treat the activated trajectories as qualitative evidence about the tool's interaction with failure cases. On \emph{fix-code-vulnerability}, CodeCanvas exhibits high alignment between analyzed regions and edits (IES=0.70; Table~\ref{tab:ies-results}) yet still fails due to a strict report-schema mismatch, illustrating that impact awareness does not substitute for contract checking. \emph{db-wal-recovery} fails across all conditions due to procedural fragility: interacting with the live database can destroy recovery evidence, a failure mode orthogonal to repository representation.

\noindent These results indicate that representation-level tools are most effective when success depends on coordinating edits across a dependency surface, but they do not replace careful contract validation or domain-specific operational discipline.

~

~

\section{Conclusion}
\label{sec:conclusion}

We introduced CodeCanvas, an MCP tool designed to reduce side-effect blindness in code-editing agents by providing blast-radius visualization and persistent reasoning state. In an exploratory evaluation on Terminal-Bench~2.0, we compared three observation modalities: text-only, graph-based (LocAgent), and visual codemaps (CodeCanvas), while holding the agent harness constant.

~

~

~

\paragraph{Key Findings.}
In this run set (7 tasks, one run per task and condition), the CodeCanvas condition solved 4/7 tasks, while Text-Only and LocAgent solved 3/7. Three observations stand out:
\begin{enumerate}
    \item \textbf{The CodeCanvas condition shows lower revision churn in this run set.} In these runs, the CodeCanvas condition exhibits lower backtracking and looping than both baselines.
    \item \textbf{Procedurally fragile tasks dominate the failure set.} Two tasks (sanitize-git-repo, db-wal-recovery) failed across all conditions, indicating that operational constraints and domain expertise can dominate representation choices.
    \item \textbf{Impact awareness does not replace contract checking.} In fix-code-vulnerability (a failure), CodeCanvas showed high alignment between analyzed regions and edits (IES=0.70) but still failed due to a strict report schema mismatch.
\end{enumerate}

\paragraph{Limitations.}~\\[0.3em]
\textbf{Study design:} Our evaluation uses $n=7$ tasks with a single run per condition, precluding strong statistical conclusions. We evaluate on one model (Claude Sonnet 4.5); generalization to other LLMs is untested. The Informed Editing Score (IES) was computed on 2 runs where CodeCanvas state was recorded, and should be treated as preliminary.

\noindent\textbf{Tool activation:} CodeCanvas' SessionStart hook is gated by a repository-size heuristic (minimum number of recognized code files). As a result, CodeCanvas state was only recorded on 2/7 tasks in this run set, and CodeCanvas-specific process metrics have limited coverage; we do not attribute aggregate performance differences to codemaps given this activation gap.

\noindent\textbf{Approach:} Blast radius quality depends on LSP support for the target language. Agents must effectively process visual or structured representations. Benefits are most pronounced for coordination-heavy tasks; discovery-heavy tasks (e.g., exhaustive vulnerability enumeration) may not benefit.

\paragraph{Future Work.}

\noindent Future work could address:
\begin{itemize}
    \item Richer codemap semantics (data flow, control flow beyond call graphs)
    \item Cross-language dependency tracking for polyglot repositories
    \item Adaptive hook policies that trigger analysis selectively based on task context
\end{itemize}

\newpage
{
    \small
    \bibliographystyle{ieeenat_fullname}
    \bibliography{references}
}

% WARNING: do not forget to delete the supplementary pages from your submission 
\clearpage
\setcounter{page}{1}
\maketitlesupplementary

\section{Why these fusion choices and not KV edits}
We avoid unproven, hard-to-stabilize KV-cache surgery. \emph{Control-tokens} are simple and transparent; \emph{gated cross-attention} is a documented, widely reproduced method for conditioning frozen LMs on auxiliary streams. Both admit clean ablations and minimize confounds from planner retraining \citep{Alayrac2022Flamingo,Hu2021LoRA,Li2021PrefixTuning}.

\section{Ground-truthing endpoints without pixels}
For \emph{AdSERP}, we use fixation points and SERP AOIs released with the dataset to build endpoint labels and to define positive/negative windows for imminence. For \emph{RecGaze}, we use interaction logs and gaze to derive endpoints for carousel targets. For \emph{OSWorld}, we use click outcomes and the verified harness to align clicks with UI state transitions; no DOM parsing is required \citep{Latifzadeh2025AdSERP,RecGaze2025,OSWorldVerified2025}.



\end{document}
